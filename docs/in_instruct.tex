\documentclass[a4paper,12pt]{article}
\usepackage{hyperref}
\usepackage{listings}
\usepackage{xcolor}
\usepackage{graphicx}

\definecolor{codeblue}{rgb}{0.1,0.1,0.7}
\definecolor{codegreen}{rgb}{0,0.6,0}
\definecolor{codegray}{rgb}{0.5,0.5,0.5}
\definecolor{codepurple}{rgb}{0.58,0,0.82}
\definecolor{backcolour}{rgb}{0.95,0.95,0.95}

\lstdefinestyle{codestyle}{
    backgroundcolor=\color{backcolour},
    commentstyle=\color{codegreen},
    keywordstyle=\color{codeblue},
    numberstyle=\tiny\color{codegray},
    stringstyle=\color{codepurple},
    basicstyle=\ttfamily\footnotesize,
    numbers=left,
    tabsize=4
}

\lstset{style=codestyle}

\title{\textbf{Installation Guide for Fuse}}
\author{}
\date{}

\begin{document}

\maketitle

\section{Introduction}
This document provides a step-by-step guide to install and set up Fuse. The installation process uses a Makefile for building the source code and an installation script \texttt{x.sh} for configuration and setup.

\section{Prerequisites}
Ensure the following tools and dependencies are installed on your system:
\begin{itemize}
	\item \texttt{gcc} or any C compiler
	\item \texttt{make}
	\item \texttt{pdflatex}
	\item Root (\texttt{sudo}) access for installation
\end{itemize}

\section{Quick Install}
To quickly configure, compile, and install Fuse, you can simply run:
\begin{lstlisting}[language=bash]
make && make install
\end{lstlisting}
\textbf{Warning:} Avoid running \texttt{make install} directly with \texttt{sudo}. This may cause files in your user directory to require root access, leading to potential permission issues.

This command will:
\begin{itemize}
	\item Automatically configure the project.
	\item Compile the source code.
	\item Install the program system-wide.
\end{itemize}

\section{Installation Process}

\subsection{Configure the Project}
Run the provided installation script to configure the project:
\begin{lstlisting}[language=bash]
./x.sh
\end{lstlisting}
This script sets up essential configuration files like \texttt{install\_config.ini} and validates the setup environment.

\subsection{Build Fuse}
Compile Fuse using the Makefile by running the command:
\begin{lstlisting}[language=bash]
make compile
\end{lstlisting}
This will:
\begin{itemize}
	\item Build all source files from the \texttt{src/} and \texttt{include/src/} directories.
	\item Generate the binary in the \texttt{build/} directory.
	\item Compile docs and put them into \texttt{build/docs/} directory
\end{itemize}

\subsubsection{Generate Documentation}
If you want to generate only documentation, use:
\begin{lstlisting}[language=bash]
make docs
\end{lstlisting}
This creates PDF documents for user manuals and installation instructions in the \texttt{build/docs} directory.
The following PDF documents are generated during the build process:
\begin{itemize}
	\item \textbf{user\_man.pdf}: The main interpreter manual for using Fuse.
	\item \textbf{in\_instruct.pdf}: Detailed installation instructions for setting up Fuse.
	\item \textbf{an\_instruct.pdf}: A manual for analyzing logs generated by the system.
	\item \textbf{LICENSE.pdf}: A copy of the LICENSE file in PDF format.
\end{itemize}

\newpage

If you need to compile only one specific document, use one of the following commands:
\begin{lstlisting}[language=bash]
make user_man
make in_instruct
make an_instruct
make LICENSE
\end{lstlisting}
This allows for selective compilation of documentation as needed.

\subsection{Install Fuse}
Run the following command to install Fuse system-wide:
\begin{lstlisting}[language=bash]
make install
\end{lstlisting}
\textbf{Warning:} Avoid running \texttt{make install} directly with \texttt{sudo}. This may cause files in your user directory to require root access, leading to potential permission issues.
\newline
This step:
\begin{itemize}
	\item Copies configuration files to \texttt{/etc/fuse/}.
	\item Sets up directories for logs, temporary files, and certificates.
	\item Validates the certificate file \texttt{certificate.txt}.
\end{itemize}

\section{Validation Checks}
During the installation, the following validations are performed:
\begin{itemize}
	\item Ensure the C and LaTeX compilers are installed.
	\item Validate the configuration and certificate files.
	\item Verify ownership and permissions of critical files.
\end{itemize}

\newpage

\section{Cleaning}
To clean up generated files and artifacts, use the following commands:
\begin{itemize}
	\item Clean temporary files:
	      \begin{lstlisting}[language=bash]
make clean_tmp
	      \end{lstlisting}
	\item Clean configuration file:
	      \begin{lstlisting}[language=bash]
make clean_config
	      \end{lstlisting}
	\item Clean temporary certificate file:
	      \begin{lstlisting}[language=bash]
make clean_certificate
	      \end{lstlisting}
	\item Clean compiled binary and documentation:
	      \begin{lstlisting}[language=bash]
make clean_compile
	      \end{lstlisting}
	\item Clean all artifacts:
	      \begin{lstlisting}[language=bash]
make clean_all
	      \end{lstlisting}
\end{itemize}

\section{Uninstallation Process}
To uninstall Fuse and remove related files, use the following commands:

\begin{itemize}
	\item Uninstall all installed files except for saved data:
	      \begin{lstlisting}[language=bash]
make uninstall
          \end{lstlisting}
	      This removes binaries installed system-wide and configuration files from \texttt{/etc/fuse/}.

	\item Remove installed files and related log files:
	      \begin{lstlisting}[language=bash]
make remove
          \end{lstlisting}
	\item Remove all installed files, logs, and saved data:
	      \begin{lstlisting}[language=bash]
make remove_all
          \end{lstlisting}
\end{itemize}
These commands run bash scripts from \texttt{/etc/fuse} with same name is target. You can delete source code of this project but still be capable of uninstalling Fuse.

\section{Troubleshooting}
\subsection{Compilation Errors}
Ensure the necessary compiler and tools are installed. Verify paths are set correctly during execution of installation script.

\subsection{Permission Errors}
Make sure you have root access when running the \texttt{make install} command.

\section{Contact Information}
For further assistance, contact me from my github profile, where Fuse is situated.

\vspace{\fill}

\begin{center}Created in \raisebox{-0.5ex}{\includegraphics[height=1em]{assets/linux.png}} \& \raisebox{-0.5ex}{\includegraphics[height=1em]{assets/neovim.png}}
	with \raisebox{-0.5ex}{\includegraphics[height=1em]{assets/love.png}}\end{center}


\end{document}

